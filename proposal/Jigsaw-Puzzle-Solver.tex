\documentclass{article}

% if you need to pass options to natbib, use, e.g.:
% \PassOptionsToPackage{numbers, compress}{natbib}
% before loading nips_2017
%
% to avoid loading the natbib package, add option nonatbib:
% \usepackage[nonatbib]{nips_2017}

\usepackage[final]{nips_2017}

% to compile a camera-ready version, add the [final] option, e.g.:
% \usepackage[final]{nips_2017}

\usepackage[utf8]{inputenc} % allow utf-8 input
\usepackage[T1]{fontenc}    % use 8-bit T1 fonts
\usepackage{hyperref}       % hyperlinks
\usepackage{url}            % simple URL typesetting
\usepackage{booktabs}       % professional-quality tables
\usepackage{amsfonts}       % blackboard math symbols
\usepackage{nicefrac}       % compact symbols for 1/2, etc.
\usepackage{microtype}      % microtypography

\title{CNN-Based Jigsaw Puzzles Solver}

% The \author macro works with any number of authors. There are two
% commands used to separate the names and addresses of multiple
% authors: \And and \AND.
%
% Using \And between authors leaves it to LaTeX to determine where to
% break the lines. Using \AND forces a line break at that point. So,
% if LaTeX puts 3 of 4 authors names on the first line, and the last
% on the second line, try using \AND instead of \And before the third
% author name.

\author{Tianbao\\
  Department of Computer Science\\
  University of Toronto\\
  Toronto, ON M5S 1A1 \\
  \texttt{tianbao@cs.toronto.edu} \\
  %% examples of more authors
  %% \And
  %% Coauthor \\
  %% Affiliation \\
  %% Address \\
  %% \texttt{email} \\
  %% \AND
  %% Coauthor \\
  %% Affiliation \\
  %% Address \\
  %% \texttt{email} \\
  %% \And
  %% Coauthor \\
  %% Affiliation \\
  %% Address \\
  %% \texttt{email} \\
  %% \And
  %% Coauthor \\
  %% Affiliation \\
  %% Address \\
  %% \texttt{email} \\
}

\begin{document}
% \nipsfinalcopy is no longer used

\maketitle

\begin{abstract}
  This paper presents an innotiative way to solve jigsaw puzzle with the help of deep concolutional neural networks. For a simpler processure, we turn the puzzle to a prediction that whether two arbitrary pieces should be adjacent. Compared with traditional solutions, using the feature maps generated from CNN can give a deeper intuition on the correlation between edges, which can improve the puzzle solution.
\end{abstract}

\section{Introduction}

Jigsaw puzzles were first introduced around 1760 for map research and then became a popular intelligence entertainment \cite{freeman1964apictorial}. The origin image is divied into $N\times M$
People usually take advantage of image information as edges, such as color, texure, instance, etc., to reconstruct the origin image. However, this problem ahas be proven to be a NP-complete one \cite{altman1989solving,demaine2007jigsaw}.

\subsection{Problem Definition}

Jigsaw puzzles aims to reshape the $N\times M$ non-overlapping, equal-sized pieces from the origin image to the right arrangement. For the worst case, it takes the complicity of $O((N\times M)!)$.

\subsection{Related Work}

Jigsaw has been researched on for many years. One most basic idea is to evaluate the compability of the adjacent pices and utake a strategy, such as greedy search, to arrange the pieces. One famous work is the Genetic Algorithm (GA) \cite{sholomon2013genetic}. Give initial candidate solutions, oit applies operations like selection, reproduction and mutation based on the color-distance fitness. Similarly but innovatively, \cite{sholomon2016dnn} first introduces deep neural network to jigsaw solver and transforms the puzzle to the piece pair adjacency prediction. It samples piece edges can learn the adjacent likelihood through DNN based on the color distance. For these two solvers, the only use color information to judge whether two pieces should be together. However, human use some others like texture to solve. So, there is still some further steps on it.

Recently, with the prosper of convolutional neural networks in computer vision area, it is also a good tool to solve jigsaw puzzles. CNN can extract regional features in many persperctives, such as color, texture, pattern, instance. These can be good inference for adjacent pieces. So, \cite{deryneural,noroozi2016unsupervised} choose to use feature maps from pre-trained CFN \cite{noroozi2016unsupervised} (siamese-ennead AlexNet \cite{krizhevsky2012imagenet}), VGG \cite{he2016deep} or Resnet \cite{simonyan2014very} and predict the location. However, the main problem for these approach is that they hold a siamese structure with shared weights for each location, which means they can only solve a limited number for pieces ($3\times3$ in \cite{noroozi2016unsupervised}, $2\times2$ and $2\times3$ in \cite{deryneural}). With the piece amount increasing, the network becomes extremely to train.

\bibliographystyle{plain}
\bibliography{Jigsaw-Puzzle-Solver}

\end{document}
